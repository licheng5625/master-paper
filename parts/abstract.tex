
\chapter{Abstract}
 
In this thesis, we propose a approach of identifying rumors in Twitter. 

Titter is a mircoblogiging service that which are used by millions users. Users can publish and exchange information with short tweets whatever when and where. This makes it a ideal media for spreading breaking news and false rumors.  

So automatic detecting rumors on social media has become a trending topic. But early researches mostly focused on rumors during one or several events like earthquake or terrorist attack \cite{oh2010exploration} \cite{tanaka2012transmission}\cite{starbird2014rumors}. But in our work, we study on the general rumors.

And most of previous work for rumor detection modeled on static features like the content of tweets or propagation features, but they ignored that those features can change during the information's propagation over time.

We build up a system with random forest and the Dynamic Series-Time Structure (DSTS) \cite{ma2015detect} with the temporal features and their varieties over time. And we test the Spike Model \cite{kwon2013prominent}, SIS Model and SEIZ Model  \cite{jin2013epidemiological} as temporal features in our model. To improve the time series model's performance at early stage of the event we develop a single tweet's credibility scoring model with zhou's CNN+LSTM model \cite{zhou2015c} which can predict a single tweet whether is rumor related or not with 81.20\%accuracy. Finally we got the accuracy of the time series rumor detecting model get 91\% accuracy within 48 hours. 

 Our experiments using the events from Twitter and our model demonstrates better performance on detecting rumors.



