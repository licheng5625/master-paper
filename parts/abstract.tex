\chapter{Abstract}
In this thesis, we tackle the issue of scholarly access to web archives. Through a combination of surveys and user studies we identified problems faced by scholars when utilizing web archives for their research. A major concern is the performance of keyword search algorithms which do not take into account a scholar's information intent which we identify as historical in nature, i.e, the user is interested in the development of a topic over time. Keyword queries with an inherent historical intent over longitudinal text corpora are interesting for a variety of special user groups like historians, social scientists and journalists. While searching articles published over time, a key preference is to retrieve documents which are from the important aspects from important points in time. To this extent, we introduce the notion of a \emph{Historical Query Intent} and define an aspect-time diversification problem over news archives. We also propose a new metric \textsc{Tia-Sbr} to evaluate the effectiveness of methods intending to solve it. 

\vspace{2mm}

We present a novel algorithm, \textsc{HistDiv}, that explicitly models the aspects and important time windows to which the results of the query belong. We test our methods by constructing a test collection based on \emph{The New York Times Collection} with a workload of 30 queries assessed manually. Our experiments show that we outperform all the competitors in most of the measures, and remain competitive in a select few.

