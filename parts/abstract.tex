\chapter{Abstract}
In this thesis, we propose a approach of identifying rumors in Twitter. 

Titter is a mircoblogiging service that which are used by millions users. Users can publish and exchange information with short tweets whatever when and where. This makes it a ideal media for spreading breaking news and false rumors.  

So automatic detecting rumors on social media has become a trending topic. But early researches mostly focused on rumors during one or several (??) events like earthquake or terrorist attack. But in our work, we more focus on general rumors.

And most of previous work for rumor detection focused on static features like the content of tweets or propagation features, and they ignored that those features change during the information's propagation over time.

we use Dynamic Series-Time Structure (DSTS)(wenxian) to capture the temporal features. And we add Spike Model, SIS Model and SEIZ Model as time series features. To improve the time series model's performance at early stage of the event we develop a single tweet's credibility scoring model which only using features which can be extracted from single tweet on the Twitter interface.


 Our experiments using the events from Twitter and our model demonstrates better performance on detecting rumors.



