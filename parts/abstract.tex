
\chapter{Abstract}
 
In this thesis, we propose a approach of detection of rumors in Twitter. 

Twitter is a microblogging service that which is already used by millions users. Users can publish and exchange information with short tweets whenever and wherever. This advantage makes Twitter an ideal media for spreading breaking news and false rumors.  

So automatic detecting rumors on social media has become a trending topic. But early researches mostly focused on rumors during one or several events like earthquake or terrorist attack \cite{oh2010exploration} \cite{tanaka2012transmission}\cite{starbird2014rumors}. But in our work, we study on the general rumors.

And most previous work for rumor detection modeled on static features like the content of tweets or propagation features, but they ignored that those features can change during the information's propagation over time.

We build up a system with random forest and the Dynamic Series-Time Structure (DSTS) \cite{ma2015detect} using the temporal features and their varieties over time. And we test the Spike Model \cite{kwon2013prominent}, SIS Model and SEIZ Model  \cite{jin2013epidemiological} as temporal features in our model. To improve the time series model's performance at early stage of the event, we develop a single tweet's credibility scoring model with zhou's CNN+LSTM model \cite{zhou2015c} which can predict a single tweet whether is rumor related or not with 81.20\% accuracy. Finally our time series rumor detecting model gets 90\% accuracy within 48 hours. 

As far as we know, we are the first research about the performances of features changing over time like the sentiment features are useless after 25 hours.    


