\chapter{Conclusions}
\label{chap:conclusions}

% In this work, we have addressed three problems toward unfolding the
% full potential of web archives to make them truly valuable
% resources. In detail:
\section{Summary of Results}
 Supporting workloads which combine text (keywords and phrases) and time are useful in many interesting search, mining, and exploration tasks over web archives. In this work, we have addressed three problems in indexing text to support such workloads in web archives. 

 We presented a novel index-organization scheme called \emph{index sharding} to process \emph{time-travel} text queries that partitions each posting list with almost zero increase in index size. Our approach is based on avoiding access to irrelevant postings by exploiting the geometry of the valid-time intervals associated with the document versions. We proposed an optimal algorithm to completely avoid access to irrelevant postings. We further fine tuned the index, taking into account the index-access costs, by allowing for a few wasted sequential accesses while gaining significantly by reducing the number of random accesses. Finally, we proposed an incremental index sharding approach that supports efficient index maintenance for dynamic updates to the index without compromising the query performance. We empirically established the effectiveness of our sharding scheme with experiments over the revision history of the English Wikipedia, and an archive of U.K. governmental web sites. Our results demonstrate the feasibility of faster time-travel query processing with no space overhead. Moreover, we showed that maintaining our index structure incrementally has large benefits over indexes which are recomputed periodically.

Next, we looked at the problem of query optimization in time-travel text search. We presented approaches for \emph{efficient approximate} processing of time-travel queries over a vertically-partitioned inverted index. By using a small synopsis for each partition we identified partitions that maximize the result size, and schedule them for processing early on. Our approach aims to balance the estimated gains in the result recall against required index-access cost. Our experiments with three diverse, large-scale text archives -- the English Wikipedia revision history, the New York Times collection and, the UKGOV dataset -- show that our proposed approach can provide close to 80\% result recall even when only about half the index is allowed to be read.

Finally, we proposed indexing and query-optimization approaches to efficiently answer \emph{phrase queries}. We considered an augmented inverted index that indexes selected variable-length multi-word sequences in addition to single words. We studied how arbitrary phrase queries can be processed efficiently on such an augmented inverted index. Moreover, we developed methods to select multi-word sequences to be indexed so as to optimize query-processing cost while keeping index size within a user-specified space budget, taking into account characteristics of both the workload and the document collection. We demonstrated experimentally the efficiency and effectiveness of our methods on two real-world corpora, i.e, the New York Times collection and the ClueWeb09 dataset.

\section{Outlook on Future Directions}

The problems addressed in this work are some of the many efficiency issues which arise in the context of text search and mining for web archives. Hence, there are many opportunities for future research. 

\paragraph{Exploiting Redundancy for Better Retrieval} Web archives are characterized by a lot of redundant content. Content is continuously added to such collections, but the addition of new content does not necessarily contribute novel content. Much of the content is either copied, enriched or recompiled from existing documents. Redundancy in text collections, apart from wasting storage space, degrades search results. Initial attempts have been made to remove redundancy in web archives by giving user the flexibility to define her notion of redundancy~\cite{paudel2013}. However, there are challenges in identifying and removing redundancy from search results. Many versions of the same document are likely to match the query because (i) either they are near duplicates or (ii) they share the context of the query terms. Designing retrieval models and indexing methods to counter the effect of such redundancy in search results, specifically for web archives, is an interesting direction for future research.

\paragraph{Phrase Indexing for Batched Query Processing} In our work, we indexed commonly occurring word sequences to improve query processing efficiency of a given phrase query. However, there are scenarios when multiple phrase queries are issued in a batch. Consider a retrieval task of finding all documents which contain mentions of the entity \texttt{Barack\_Obama}. Assume that we already know the different textual representations or labels used to denote this entity, i.e., \kwquery{president of U.S.A}, \kwquery{president of the united states}, \kwquery{leader of the united states} and so forth. Treating each label as a phrase, each entity is associated with a batch of phrases as above. The research challenge is to come up with novel query processing methods, based on our current work on phrase indexing, to efficiently process such batches of phrase queries. 

\paragraph{Mining and Exploration of Web Archives} 
 Exploratory tasks over web archives require multiple rounds of searching and aggregation over both the text and time axes. A promising research direction would be to identify how users interact with such collections and what features are they interested in. As an example, consider a user interested in all entities present in the documents which are results of the time-travel query \query{google io}{03/2013 - 06/2013}. The challenges are in improving retrieval effectiveness and evaluation of such systems.


