\chapter*{Zusammenfassung}
In dieser Arbeit, schlagen wir einen Ansatz der Erkennung von Ger\"uchten in Twitter.

Twitter ist ein Microblogging-Dienst, der bereits von Millionen Benutzern verwendet wird. Benutzer k\"onnen Informationen irgendwo und irgendwann mit kurzen Tweets ver\"offentlichen und austauschen. Dies macht Twitter zu einem idealen Medium f\"ur die Verbreitung von aktuellen Nachrichten und falschen Ger\"uchten.

So automatische Erkennung von Ger\"uchten auf Social Media hat sich zu einem Trend Thema. Aber fr\"uhe Forschungen konzentrierten sich haupts\"achlich auf Ger\"uchte w\"ahrend eines oder mehrerer Notf\"allen wie Erdbeben oder terroristische Angriffe \ zitieren {oh2010exploration} \cite {tanaka2012transmission} \cite {starbird2014rumors}. Aber in unserer Arbeit studieren wir \"uber die allgemeinen Ger\"uchte.

Wir erstellen ein System mit Ramdom-Forest und die Dynamische Serien-Zeit-Struktur (DSTS) \cite {ma2015detect} mit den zeitlichen Merkmalen und deren Sorten im Laufe der Zeit. Und wir testen das Spike Modell \cite {kwon2013prominent}, das SIS Modell und das SEIZ Modell \ cite {jin2013epidemiological} als zeitliche Merkmale in unserem Modell. Um die Leistungsf\"ahigkeit des Zeitreihenmodells im frühen Stadium des Ereignisses zu verbessern, entwickeln wir ein Modell f\"ur die Glaubw\"urdigkeit einzelnes Tweets von Zhou mit Zhou's CNN + LSTM-Modell \cite {zhou2015c}, das einen einzelnen Tweet vorhersagen kann, ob es sich um ein Ger\"ucht oder nicht um eine Genauigkeit von 81,20\% handelt. Schlie\ss lich haben wir die Genauigkeit der Zeitreihe Ger\"ucht Erkennung Modell erhalten 90\% Genauigkeit innerhalb von 48 Stunden.  

Soweit wir wissen, sind wir die erste Forschung \"uber die Leistungsf\"ahigkeit von Merkmalen im Laufe der Zeit wie die Sentiment Features sind nutzlos nach 25 Stunden.