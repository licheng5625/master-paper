\chapter*{Zusammenfassung}
%
%Webarchive bezeichnen einerseits Archive urspr\"unglich im Web
%ver\-\"offent\-lichter Inhalte (z.~B. das Internet
%Archive~\footnote[1]{\textbf{ http://www.archive.org}}), andererseits
%Archive, die vor m\"og\-lich\-er\-wei\-se langer Zeit
%ver\"offentlichte Inhalte im Web zugreifbar machen (z.~B. das Archiv
%von The Times~\footnote[2]{\textbf{
%    http://archive.timesonline.co.uk}}). Ein gewachsenes Bewusstsein,
%dass origin\"ar digitale Inhalte (z.~B. Webseiten) bewahrenswert sind,
%sowie verbesserte Digi\-tal\-is\-ierungs\-verfahren, haben dazu
%gef\"uhrt, dass Anzahl und Umfang von Webarchiven in den letzten
%Jahren zugenommen haben. Die beiden zuvor genannten Beispiele, das
%Internet Archive und das Archiv von The Times, zeigen zwei wichtige
%Merkmale von Webarchiven: Erstens sind diese h\"au\-fig umfangreich
%und umfassen Millionen oder gar Milliarden von Dokument\-en -- seit
%1996 hat das Internet Archive 150 Milliarden Kopien von Webseiten
%gespeichert. Zweitens decken Webarchive oft einen sehr langen Zeitraum
%ab -- so stammen die \"altesten der im Archiv von The Times
%enthaltenen Artikel aus dem Jahr 1785. Bestehende Suchverfahren
%ber\"uck\-sich\-tigen diese Charakteristika von Webarchiven derzeit
%nicht. Damit Webarchive ihr volles Potenzial aussch\"opfen und zu
%wirklich wert\-vollen Wissensquellen werden k\"onnen, bedarf es
%durchdachter Suchverfahren. Die vorliegende Arbeit
%befasst sich mit den drei folgenden relevanten Teilproblemen:\\
%
%\textit{Zeitreise-Textsuche} effizient zu unterst\"utzen ist das erste
%dieser Teilprobleme. Die sogenannte Zeitreise-Textsuche erlaubt es
%Benutzern, Anfragen zu formu\-lieren, wie sie auch an Websuchmaschinen
%gestellt werden, diese jedoch zu\-s\"atz\-lich mit einer Zeit von
%Interesse (z.~B. einem Tag im vergangenen Jahr) zu versehen. Nur jener
%Teil des Webarchives, der zum genannten Zeitpunkt existiert hat, wird
%f\"ur eine solche Zeitreise-Suchanfrage betrachtet. Sucht man
%beispiels\-weise nach zeitgen\"ossischen Dokumenten zur FIFA
%Weltmeisterschaft im Jahr 2006, so kann man die Suchanfrage
%\textsf{fifa world cup}@\textit{July 2006} formu\-lieren und damit nur
%auf jenen Dokumenten suchen, die im Juli 2006 existiert haben. Wir
%beschreiben einen effizienten Ansatz zur Unter\-st\"utzung von
%Zeitreise-Suchanfragen. Uns\-er Ansatz basiert auf dem bekannten
%invertierten Index und erweitert ihn derart, dass G\"ultigkeitszeiten
%von Daten erfasst werden. Um Indizes trotz des gro{\ss}en Umfangs von
%Web\-ar\-chi\-ven kompakt zu halten, stellen wir Verfahren zur
%Verschmelzung von zu aufeinanderfolgenden Versionen des selben
%Dokumentes geh\"orenden Daten vor. Ein weiterer Beitrag sind
%Verfahren, welche eine Feinabstimmung des Index im Hinblick auf zu
%erf\"ullende Leistungsgarantien oder Speicher\-be\-schr\"an\-kung\-en
%erlauben. Die\-se Verfahren basieren auf einer zeitlichen
%Partitionierung und Repli\-zier\-ung der im Index vorhandenen
%Daten. Sowohl die Verfahren zur zeitlichen Verschmelzung als auch die
%Partitionierungsverfahren sind als Optimierungs\-pro\-bleme
%formalisiert. Wir beschreiben Algorithmen zur Berechnung von optimalen
%und --wenn sinnvoll-- an\-n\"ah\-er\-ungs\-weisen
%L\"o\-sung\-en. Anhand von umfangreichen Experimenten auf drei
%re\-pr\"a\-sen\-tativen Web\-archiven wird gezeigt, dass der
%vorgeschlagene
%Ansatz praktikabel ist.\\
%
%W\"ahrend des von einem Webarchiv abgedeckten Zeitraumes kann sich die
%g\"angige Terminologie deutlich ver\"andert haben. Diese
%\textit{Terminologie\-ver\"ande\-rung} ist urs\"achlich f\"ur eine
%sich weitende Kluft zwischen heute g\"angiger Terminologie, die von
%Benutzern verwendet wird, um Suchanfragen zu formu\-lieren, und jener
%Terminologie, in der archivierte Dokumente geschrieben wurden. F\"ur
%die Suchanfrage \textsf{saint petersburg museum} beispielsweise werden
%in den 1970ern ver\"offentlichte Dokumente, die wertvolle Information
%\"uber Museen in Lenin\-grad enthalten, oft nicht gefunden. Um einer
%daraus resultierenden Verminderung der Ergebnisg\"ute
%entgegenzuwirken, stellen wir ein Verfahren zur automatischen
%Umformulierung von Suchanfragen vor. Anhand von zeit\-ab\-h\"ang\-igen
%Sta\-tistiken \"uber das gemeinsame Auftreten von Termen, ermittelt
%unser Verfahren solche Terme, die in der Vergangenheit eine \"ahnliche
%Bedeutung hatten wie die in der vorliegen Anfrage enthaltenen
%Ter\-me. Hierzu werden zeit\-ab\-h\"angige Kontexte von Termen
%verglichen -- der Term \textsf{leningrad} beispielsweise trat in der
%Vergangenheit h\"aufig mit Termen wie \textsf{russia},
%\textsf{hermitage} und \textsf{tsar} auf, genau wie es heute der Term
%\textsf{saint petersburg} tut. Unter Verwendung eines Hidden Markov
%Modells bestimmt das Verfahren dann gute Umformulierungen der
%vorliegenden Suchanfrage, indem es solche als \"ahnlich identifizierte
%Terme zusammensetzt, welche koh\"arent (d.~h. zusammen Sinn ergebend)
%und popul\"ar (d.~h. g\"angig verwendet) sind. Hierzu greift das
%Verfahren wiederum auf zeitabh\"angige Termstatistiken zur\"uck. Wir
%beschreiben wie sich das vor\-ge\-schla\-gene Verfahren implementieren
%l\"asst, so dass interaktive Antwortzeiten
%erreicht werden, und demonstrieren seinen praktischen Nutzen.\\
%
%Beim dritten Teilproblem, das nicht nur f\"ur die Suche in Webarchiven
%von Interesse ist, handelt es sich um
%\textit{Informationsbed\"urfnisse mit deutlichem Zeitbezug}. Diese
%lassen sich h\"aufig am besten durch Dokumente bedienen, welche auf
%eine bestimmte Zeit Bezug nehmen. Ein konkretes Beispiel stellt die
%Such\-an\-frage \textsf{crusades 13th century} dar. Bestehende
%Verfahren scheitern oft an solchen Informationsbed\"urfnissen, da
%ihnen in Dokumenten enthaltene Zeitbez\"uge (z.~B. \textsf{``in
%  1202''}) und deren Bedeutung verborgen blei\-ben. Eine
%Schwier\-ig\-keit beim Umgang mit Zeitbez\"ugen ist deren inh\"arente
%Unsch\"arfe -- so kann der genannte Zeitbezug \textsf{``in 1202''}
%sowohl auf einen bestimmten Tag, aber auch auf das gesamte Jahr
%verweisen. Das vorgestellte Retrieval-Modell be\-r\"uck\-sich\-tigt
%Zeitbez\"uge, deren Bedeutung sowie die ihnen inh\"arente Unsch\"arfe
%und f\"ugt diese nahtlos in einen auf Language Models beruhenden
%Retrieval-Ansatz ein. Die zentrale Idee dieses Modelles liegt darin,
%Zeitbez\"uge als Mengen jener exakten Zeitintervalle zu
%re\-pr\"a\-sen\-tieren, auf die sie verweisen k\"onnen, und damit ihre
%Unsch\"arfe zu erfassen. Hierauf aufbauend entwerfen wir ein
%generierendes Modell f\"ur Zeitintervalle und verfolgen einen
%Query-Likelihood-Ansatz, um die Relevanz eines Dokumentes zu einer
%Suchanfrage mit Zeitbezug zu sch\"atzen. Umfangreiche Experimente auf
%zwei Do\-ku\-ment\-en\-kollektionen unter Verwendung von Suchanfragen
%und Relevanzbewertungen, die wir mittels Online-Be\-fra\-gung echter
%Benutzer gesammelt haben, zei\-gen, dass unser Ver\-fahren eine
%we\-sent\-liche Ver\-bes\-ser\-ung der Ergebnisg\"ute f\"ur die
%betrachteten Informationsbed\"urf\-nis\-se mit deutlichem Zeitbezug
%erzielt.
%
%

%%% Local Variables: 
%%% mode: latex
%%% TeX-master: "phd-thesis"
%%% End: 
