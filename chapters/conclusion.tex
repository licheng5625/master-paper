\chapter{Outlook} % (fold)
\label{cha:conclusion_and_future_work}


\section{Conclusion}
\vspace{-2mm}
With more nations working actively to preserve the web due to the importance it plays in our lives, scholars from various fields have started looking to web archives as a data source waiting to be analyzed. We find that scholarly research with web archives carried out until today can be broadly classified as link based and content based studies. The link structure of web archives has been studied on a quantitative scale by Internet research institutes and computer scientists. They either study the link structure over time to analyze the evolution of the web or try to correlate the linking structure to real world phenomena. In this thesis we first looked at the usage of web archives by scholars. According to Br\"ugger there are 4 major steps when conducting any type of research on web archives: corpus creation, analysis, dissemination and storage. An ideal system for web archive research should support all 4 steps however we find that adequate steps are only now being made to help researchers with the first step: corpus creation. Humanities scholars are usually interested in doing small scale qualitative and quantitative studies for which they need to explore web archives for relevant material. But the access to web archives with current web archive access systems like the wayback machine and rudimentary keyword search is not satisfactory. There are many other factors as well that have led to the lack of humanities studies on web archives. We attempted to illicit these through literature surveys and group discussions with humanities scholars conducting web archive research. Some of the interesting outcomes of this study were:

\begin{itemize}
	\item The absence of abilities to explore the web archive discourage scholars from pursuing their ideas. Search engines are the predominant way of exploring the web today whereas archives seem to be left behind.
	\item The wayback machine is not an effective tool for building a corpus unless you know the exact set of URLs you are interested in. Full text search should be provided for researchers who want to explore the archive.
	\item More details of the crawl should be exposed so that scholars can motivate their corpus of study better. The details of the crawl strategy should be added to the meta data of all documents.
	\item Scholars are used to working with well curated archives. Web archives pose greater burden for scholars in corpus creation since they need to curate the materials themselves.
	\item Due to the inconsistent nature of web archives, an error margin should be defined and an approximation of confidence as well even for small scale qualitative studies.
	\item Algorithms and features used to help scholars find new documents should be made transparent to users so that they can effectively document the corpus creation process.
	\item A web archive search system should allow for the corpus and its history of creation to be exported in standard formats researchers are used to working with. 
	\item Scholars are interested in leveraging digital methods to analyze their data but find it hard to find tools flexible enough for their needs.
\end{itemize}



Scholars also saw potential in the usage of web archives as a playground for combining qualitative and quantitative studies. Based on our discussions with the scholars who attended the summer school in Aarhus University, we found that many would be interested in the ability to scale results up. They also bemoaned the lack of analysis tools currently available to work with web archives. The lack of technical expertise was clearly affecting a scholar's ability to conduct his research even though they knew what kind of analysis they wanted to do. We proposed that computer scientist should bridge this gap by either providing tools to support scholars or work with them on scaling up their hypothesis using pattern recognition and data mining techniques.

Initiatives are being made kick-start humanities research in web archives but to do the necessary infrastructure to first access web archives has to be in place. These initiatives are currently being carried out by national libraries and institutions like the Internet Archive and the IIPC. The BUDDAH project, a joint effort between the British Library and leading Universities in the UK, is working towards building a web archive search system by working in tandem with humanities scholars working with the UK web archive. We held discussions with members of the BUDDAH project and scholars to identify the pitfalls with the current web archive search system they were working with. The British Library web archive search prototype already provides its users with keyword search and filtering based on domain and crawl date. As noted by Costa et. al. current IR techniques like keyword search do not fare well in web archive search. We observed this first hand with the British Library's choice to rank search results by crawl date. This makes it very difficult for users to easily comprehend a large result set. We decided to better understand a scholar's search intent by analyzing written descriptions of proposed studies on web archives by humanities scholars involved in the BUDDAH project. We found that scholars are interested in studying results over time and covering a variety of aspect for their topic. 

To this end we introduced the notion of a historical query intent over longitudinal collections like web archives. Historical query intent was modeled as a 2 dimension diversification problem where one dimension is time and the other aspects. In this thesis we limit ourselves to a subset of web archives: the news archive. We proposed a new evaluation metric \textsc{Tia-SBR} to evaluate the performance of retrieval models for this task. We also adapted existing diversity metrics to take time into account. To evaluate the retrieval models we built a new temporal test collection based on 20 years of the \emph{New York Times} collection. We strengthened existing 1 dimensional diversity methods by linearizing the aspect-time space into a single set. We also consider temporal diversity retrieval models in our experiment. We introduced \textsc{HistDiv} which shows improvements over temporal and non-temporal methods for most of the time-aware diversification methods. We also outperform all competitors in \textsc{Tia-SBR} showing the suitability of our approach for historical query intents. We observe that \textsc{HistDiv} works well for topics which have aspects that span across multiple time intervals and have fluctuating importance at different times. It trades-off nicely between important aspects and important times which we perceive as important in historical search. \textsc{HistDiv} does not perform quite as well for queries which only one dominant aspect at a certain time window. We also described the architecture of the History Search system which consists of: a REST API to access the retrieval models used in our experiments, a user interface designed specifically to show results as newspaper articles and a time line to provide an overview and ability to filter.

\section{Future Work} % (fold)
\label{sec:future_work}

The performance of the HistDiv algorithm encourages us to look beyond the traditional approaches and devise methods more suited to archives. HistDiv currently considers time as a set of disjointed windows. It will be interesting to see if bursts rather than time windows are more effective for historical query intents. In our current implementation, subtopics are mined using wikiminer which is a relatively naive approach. In the future, we want to experiment with different types of aspect mining like LDA for instance. We also want to add more dimensions to the model like geographic locations. Just like aspect utility decays based on time, we can also decay it based on location. The absolute distance between lat long coordinates can be used or a taxonomy based approach.

The results also indicate where HistDiv performs badly and other algorithms perform well. We believe that if the user is presented with a choice of algorithms, she can effectively select the algorithm best suited to her need. A step beyond this would be to train a classifier that selects the appropriate retrieval model based on features from the query's aspect and temporal profile.

The test collection we built is still relatively small when compared to collections by TREC. We must strive to add to this test collection, by increasing the workload, or explore different avenues for evaluating retrieval models meant for archive search. For our test collection it will also be interesting to get actual historians to provide topics. More research is also needed to construct test collections for other query intents for archives. 

Having performed admirably in a news archive test collection, it will be interesting to test HistDiv and the other methods on a web archive or at least a subset which is not just news. Existing web crawl datasets only span a small period of time but it might be interesting nonetheless to experiment with smaller time window sizes for these collections. Another important temporal feature that we want to consider is the temporal references within an article. It is reasonable to assume that major bursts are usually covered by summary articles after the end of the burst. These articles are valuable but may get neglected because they do not occur within the burst based on our current algorithm. HistDiv's ability to find documents from important time windows and aspects can also be applied in Temporlia's query classification task to interesting effect.

There is also much to be done to help scholars better engage with web archives. In this thesis we have only scratched the surface of the issues surrounding intelligent access to web archives. We must work together with scholars in order to develop effective algorithms and user friendly systems so that the researchers of tomorrow can conduct their studies with far greater ease than today.

% section future_work (end)

% chapter conclusion_&_future_work (end)