

6- small scale link network study to uncover online networks for the poetry society
4- study the evolution of recruiting by the MOD and the influence of social networks in recruiting no indication of scale
5- create a sample of public archaeology websites to see how it has been presented over the years. characterize the the interface elements like nav elements and also categorize the content based on some semantics.
3 - the evolution of accessibilty of websites for the disabled. did websites start adopting the standard when it was released? Web Content Accessibility Guidelines (WCAG) Also  assess the web presence of the leading disability organisations between 1996 and 2010. manually crafted dataset. history of disability.
2- parliamentary web archive. irrelevant. both quant and qual studies.
1- shoebox historians. To focus this initial case study, the theme of relevant blogs and citizen-archives would be prisoner of war diaries from the Second World War. again small scale. hard to find all shoeboxes.


*7- history of vicky price from 2004 to 2013. changes in online reputation. small datasets with keywords. documents from important and unexpected aspects. outcome: categorize the data by facets.


8- Revealing British Eurosceptism in the UK Web Domain and Archive. dataset consisting of political party websites. both qualitative and quantitative results. looking for patterns in aggregates. build corpus using search


*9- comparing the past with the present for beats literature. corpus formed based on entities(authors) How has reception to the Beats developed during the period, and has the more democratic nature of internet discussion changed scholarly and non-scholarly responses?
My case study will use the UK web archive to locate key websites or webpages including Beat-related blogs, online magazines, fansites, forums, academic websites, and comments sections. I will use proximity searches and sentiment analysis when searching, in order to assess any cultural developments or trends




10- how notions of heritage are described and utilised in discussions of these televisual descriptions and also how heritage is discussed and described away from the television context. The value of analytical access to the data in the web archive is that it will allow the identification of places where one might not expect heritage to be discussed and aid in the development of a new way of thinking about what heritage can be. 
How has heritage been discussed and understood over the past. 
keyword based approach to corpus formation


Another aspect of access to web archives is the visualisation of search results. The 10 blue links paradigm followed by most search systems today is good for the casual user but for the researcher who is looking to gather new insights from these results more can be done to help them. The simple addition of a timeline would help researchers get a better understanding of the temporal distribution of the results.