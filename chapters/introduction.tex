\chapter{Introduction} % (fold)
\label{cha:introduction}


The web today encompasses almost all walks of life. It is the medium of choice to share and access information rapidly. Every major organisation and government has an online presence. Individuals also possess strong online auras thanks to blogs and social networks. Newspapers in print are slowly losing out to their online counterparts. All major newspapers now have an online portal where they publish their stories. Major events are being covered with live blogs and social netowrks as well. The web also promotes discourse and debate through social media. Organizations and companies use their web presence to promote and deliver news to their customers. Even our shopping has moved online. 

All of the information on the web serves as valuable documentation for human society so it is natural for us to want to preserve it. Just like the digitization of old books and newspapers to preserve them for longer, society and scholars in particular are interested in preserving the web~\cite{ross2000changing}. Newspapers have a long history of being maintained in archives for future reference. \emph{The New York Times} like many other leading publications maintain their archives on the web as well. 

%The characteristic nature of print media is that its inherently read only. A newspaper or magazine cannot be rewritten once published.  

Since the mid 1990s the Internet Archive begun to preserve the web. Today, to deal with the explosion in growth of the web, the Internet Archive is joined by several national libraries and institutions. With the growing importance of the web to society, researchers from sociology, politics and history have started to delve into web archives to study human society and the web itself. Web archives present many opportunities for various kinds of historical analyses~\cite{schreibman2008companion}, cultural analyses~\cite{toyoda2003extracting}, and analytics for computational journalism~\cite{cohen2011computational}. Researchers can retrospectively analyze the web and study the development of trends ~\cite{kumar2005bursty} or the web at a certain time. Many studies of social phenomena on the web to date have been based on observations conducted at a single point in time (the present), or during a more extended period but without explicit consideration of the possibility of changes during that period. Web archives expand the scope of potential research by enabling developmental analyses of web and certain real world phenomenon as it changes over time by tracking changes in web objects between cycles of capture in the archiving process.

For scholars of any field to study web archives they need to construct corpora specific to their study \cite{palmer2008scholarship} and to do that they need to be able to access data in the archive effectively. The Internet Archive provides a look up service called the wayback machine\cite{wayback} which allows the user to enter a URL and see the various versions of the URL over time. He can also surf the web at that time using the hyperlinks. This however is an inferior access method when compared to keyword search which overtook surfing as the dominant way of accessing the web a decade ago. However based on user studies conducted with the Portugese Web Archive~\cite{pta}, it was found that the state of the art search techniques applied on the web are not satisfactory for web archive search \cite{costa2010understanding}. 

Through a series of discussions with scholars using web archive search systems we found that in particular the ranking of documents is not helpful. Scholars are interested in studying phenomena as they develop across time; more specifically, they are interested in the history of a topic. History implies that scholars would like to study documents cover different aspects of the topic from important time periods. For any longitudinal collection, like archives in general, with standard retrieval models users are not guaranteed to get documents from across time and also covering different aspects within a reasonable top $k$ documents. And for scholars who pay very close attention to every document in the result set they may have to go through the entire search result set which, after filtering, can be over a few thousand at least to find what they are looking for. Instead, if users are presented with documents that cover important historical aspects at the top of the result list, they can quickly get a grasp of the important time intervals and aspects of their topic. From here they can choose a number of ways to reformulate their query if need be to unearth details about a particular time window or aspect.

In our work we attempt to answer the historical search intent of scholars over a longitudinal collection. We conduct our experiments on news archives because newspaper articles encode history as it happens by capturing events and their immediate impact on society, politics, business and other important spheres. These are of immense value to historians, sociologists, and journalists who rely on a fairly reliable, accurate and time-aligned information sources. Consequently, numerous news archives like~\cite{nytarchive,caliarchive,ukarchive} have gained prominence consisting not only of digitally born content but digitized content from the distant past and can be considered as a subset of the overarching web archive. Query models and indexing methods based on keyword search and time filtering~\cite{berberich_time_2007,anand_2012_index} have been proposed which are a natural access methods to discover and explore content in such archives. However the support for advanced retrieval models which encode historical intents is limited. 

The classical diversification approaches like~\cite{carbonell1998use,santos2010exploiting,agrawal_diversifying_2009,Carterette:2009:PMR:1645953.1646116,dang_diversity_2012} which are optimized for the more traditional query intent do not take time into account. As a result, documents retrieved might still cover a good number of aspects but might be from the same time period disregarding the temporal salience of the aspect. Also, time-aware approaches which take into account latent topics or aspects like~\cite{ecir/NguyenK14} are optimized to present results which are valid at querying time or in other words reward recency. On the other hand Berberich et. al. in~\cite{lm+t+d} diversify based on time without explicitly considering document aspects. Although this ensures that results are temporally distant from each other, as a consequence of the inherent aspect-agnostic nature they still might belong to similar aspects. We on the contrary, explicitly model document aspects and jointly diversify both in the aspect and time dimensions. 

In this thesis, we make the following contributions:

\begin{itemize}
	
	\item Based on a user study, we identify the problems faced by humanities scholars like historians, sociologists, and journalists when engaging with web archives for their research. In particular, we found that retrieval models currently available are not satisfactory for a scholar's search intent of finding documents to study the past of a topic.

 	\item we introduce the notion of \emph{Historical Query Intents} and model this as a search result diversification task on both the aspect and time dimensions.

 	\item we develop a novel retrieval algorithm called \textsc{HistDiv} which jointly diversifies both dimensions by appropriately discounting the contribution of aspect mass and time. we also propose a evaluation measure called \textsc{Tia-SBR}.

 	\item and finally we establish the effectiveness of our methods by building a test collection based on the 20 years of the \emph{New York Times Collection} as a dataset and a workload of 30 manually judged queries.
\end{itemize}

The rest of this these is organized as follows: In Chapter~\ref{cha:researcher_engagement_with_web_archives} we try to understand the research practices of humanities scholars when working with web archives. We also seek to find problems faced by the scholars when accessing web archives using keyword search. Based on our findings we introduce Historical Query Intents(HQIs) in Chapter~\ref{cha:historical_search}. In the same chapter we also motivate and develop a novel retrieval model suited for HQIs. The experiments to test the performance of this model against its competitors is covered in the same chapter. The penultimate Chapter~\ref{cha:the_historical_search_system} details the architecture of the system that was built to demonstrate the efficacy of the retrieval models to users. Finally, in Chapter~\ref{cha:conclusion_and_future_work} we add some concluding remarks and describe future work.

% chapter introduction (end)