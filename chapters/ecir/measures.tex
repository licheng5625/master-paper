\section{Time-Aware Effectiveness Measures} 
\label{sec:measures}
 
 Since we follow a diversification based approach to historical search we adapt the existing diversity measures to take time into account. We first extend existing diversification metrics by making them two dimensional, and then we proceed to propose a new metric which takes into account the importance of the aspects along with time.

%As with any user intent for search, using the right collection to evaluate the performance of retrieval models is imperative. According to Soboroff \cite{soboroff2013building}, to build a test collection the following points need to be addressed:

%\begin{enumerate}
%	\item Determine the task.
%  \item Identify a document collection.
%  \item Build topics.
%  \item Make relevance judgments.
%\end{enumerate}

\subsection{Time-augmented Diversification Metrics} % (fold)
\label{sub:diversification_metrics}

Similar to aggregating metrics like \textsc{Precision, NDCG, ERR} and \textsc{Subtopic Retrieval} across intents in~\cite{santos_intent-aware_2011}, we aggregate the contribution of each of these metrics to every time window. Let $C$ is the set of subtopics for a topic $q$ in the workload $Q$ and $rel(j|c)$ is the relevance of a document at $j$ to a subtopic $c$ belonging to the time window $t$. In this subsection we show how th traditional intent aware measures can be augmented to include time.


% subsection diversification_metrics (end)

%Like the TREC diversification task we also define our own measures to discriminate between historical search retrieval models. Since we follow a diversification based approach to historical search we adapt the existing diversity measures to take time into account. 

%For the diversity track, traditional IR measures were adpated by adding a dimension of intents to evaluate the performance of retrieval models at diversifying said intents. For example, precision was modified to intent aware precision by computing a weighted summation of precision given a particular intent, over all intents. 

% \begin{equation}
% IA-Precsion_k=\frac{1}{|C|}\sum_{c=1}^{|C|}\frac{1}{k}\sum_{j=1}^{k}rel(j|c)
% \end{equation}

% where $C$ is the set of subtopics for a topic $q$ and $rel(j|c)$ is the relevance of a document at $j$ to a subtopic $c$. Similarly 

% we extend the intent aware measures by adding time awareness to make them more suitable at measuring the performance of aspect and time diversification retrieval models. Time-aware Intent-aware Precision (TIA-Precision) is defined as follows:

% \begin{equation}
% TIA-Precsion_k= \sum_{t \in T(q)} P(t|q) \: \frac{1}{|C|}\sum_{c=1}^{|C|}\frac{1}{k}\sum_{j=1}^{k}rel(j|c,t)
% \end{equation}

\paragraph{Time aware alpha-IA-NDCG}


NDCG (Normalized Discounted Cumulative Gain) is a measure that rewards result lists of length $k$ that rank relevant documents closer to the top of the list. The core of the measure is cumulative gain which is then discounted based on the position of the documents in the ranked result list. Intent awareness is added to the measure by computing a weighted sum of NDCG for each subtopic $c$ such that $\sum_{c} P(c|q) = 1$. We introduce time awareness in a similar fashion. The importance of a time period or window is represented by conditional probability of a time window over the query. We compute intent aware NDCG over each time window within the time range of the collection and compute the weighted sum over all time windows where $\sum_{t} P(t|q) = 1$. Time aware Intent aware $\alpha$-NDCG rewards result lists of length $k$ that rank, for each intent, relevant documents from important time windows closer to the top of the list.
\begin{equation}
P(t|q) =  \frac {|d_{t,q}|}{|d_{q}|}  
\end{equation}

where $d_{t,q}$ is a document with time stamp $t$ and relevant to the query $q$.

\begin{equation}
DCG = \sum_{j=1}^{k} \frac {2^{rel(j)} -1}{log(1+j)}  
\end{equation}


where $j$ is the rank of the document and $rel(j)$ is the binary relevance judgement of the document at j. For intent awareness, DCG is modified by changing $rel(j)$ to $rel(j|c)$. NDCG is computed as $\frac{DCG_{k}}{Ideal DCG_{k}}$.

\begin{equation}
\textsc{ia-Ndcg(Q)}_k = \sum_{c} P(c|q) NDCG(Q, k|c)
\end{equation}

\begin{equation}
\textsc{Tia-Ndcg(Q)}_k = \sum_{t} P(t|q) \sum_{c} P(c|q) NDCG(Q, k|c)
\end{equation}



\paragraph{Time aware IA-ERR}


Intent aware estimated reciprocal rank (IA-ERR) has been used by TREC as its primary measure for measuring diversity performance. It is a cascade user model based metric and it is shown to be more accurate than position based metrics like NDCG. Temporal ERR-IA is computed by introducing time as an extra dimension of intent in the original computation of ERR-IA. 

\begin{equation}
\textsc{ia-Err}_k = \sum_{r}^{n} \frac{1}{r} \sum_{c} P(c|q) rel(r|c) \prod_{i=1}^{r-1} (1-rel(i|c))
\end{equation}

\begin{equation}
\textsc{Tia-Err}_k = \sum_{r}^{n} \frac{1}{r} \sum_{t} P(t|q) \sum_{c} P(c|q,t) rel(r|c) \prod_{i=1}^{r-1} (1-rel(i|c))
\end{equation}


%This is what it should be: - 
%\begin{equation}
%2d\:ERR-IA = \sum_{r}^{n} \frac{1}{r} \sum_{t} P(t|q) \sum_{c} P(c|q,t) rel(r|c,t) \prod_{i=1}^{r-1} (1-rel(i|c,t))
%\end{equation}
%for that measure we need a 2 level relevance judgement or we model the burst size as relevance judgements somehow.

\paragraph{Time aware IA precision}

Time-aware intent-aware precision at $k$, \textsc{Tia-Precision}$@k$ can be defined as  

\begin{equation}
\textsc{ia-Precision}_k=\frac{1}{|C|}\sum_{c=1}^{|C|}\frac{1}{k}\sum_{j=1}^{k}rel(j|c)
\end{equation}

where $C$ is the topic represented as a set of subtopics $c$. $rel(j|c)$ is the relevance of a document at $j$ to a subtopic $c$. This measure determines how precise a ranked list is over a set of intents. For historical search we need to measure precision not only over subtopics but also over time. We want a ranked list to consist of documents from diverse subtopics and diverse time windows. To measure the precision of covering time as well as subtopics we introduce time in IA-Precision in the following way:

\begin{equation}
\textsc{Tia-Precision}_k= \sum_{t \in T(q)} P(t|q) \: \underbrace{\frac{1}{|C|}\sum_{c=1}^{|C|}\frac{1}{k}\sum_{j=1}^{k}rel(j|c,t)}_{\textsc{Precision}\emph{ in time wind. } t}
\end{equation}

\paragraph{Time aware average IA-precision \& MAP-IA}

Average IA-Precision is used to measure how precise a retrieval model is for $1 \leq k \geq n$. To compute Mean average precision, Average IA-Precision is computed for all queries in Q and then we compute the mean. We use Time Aware IA-Precision instead of the standard IA-Precision to introduce time awareness.

\begin{equation}
IA-AvgP(q)_k = \frac{\sum_{k=1}^{n} (2d-IA-Precision(k) * rel(k))}{\# relevant\:documents} 
\end{equation}

\begin{equation}
  \textsc{Tia-MAP}_k = \frac{\sum_Q IA-AvgP(q)}{|Q|}
\end{equation}


\paragraph{Subtopic recall}
Subtopic recall is the measure of intent coverage for a given result list at depth k. 
\begin{equation}
\textsc{Sbr}_k =\frac{ |subtopics(S)|}{|subtopics(q)|}
\end{equation}

\begin{equation}
\textsc{T-Sbr}_k =\frac{ |subtopics(S)|\:  +\:  |timeWindows(S)| }{|subtopics(q)|\:  +\:  |timeWindows(q)|}
\end{equation}

\textsc{T-Sbr}$_k$, our adaptation of subtopic recall, considers both time windows and subtopics to be members of a single set of intents. $subtopics(S)$ and $timeWindows(S)$  denote set of subtopics and time windows covered by the diversified result set $S$ respectively. Similarly,  $subtopics(q)$ and $timeWindows(q)$ denote the set of all subtopics and time windows for a given query $q$.


% \begin{equation}
% \textsc{Tia-NDCG}_k = \sum_{t} P(t|q) \textsc{NDCG}_{k,t}
% \end{equation}

% \begin{equation}
% \textsc{Tia-ERR} = \sum_{r}^{n} \frac{1}{r} \sum_{t} P(t|q) \sum_{c} P(c|q) rel(r|c,t) \prod_{i=1}^{r-1} (1-rel(i|c,t))
% \end{equation}

% \begin{equation}
% \textsc{Tia-AvgPrecision}_k = \frac{\sum_{k=1}^{n} (\textsc{Tia-Precision}_k * rel(k))}{|R|} 
% \end{equation}

% \begin{equation}
% \textsc{Tia-MAP} = \frac{\sum_Q \textsc{Tia-AvgPrecision}}{|Q|}
% \end{equation}

% \begin{equation}
% \textsc{SBR}_k = \frac{\# subtopics \: covered \: in \: k\:  +\:  \# time \: windows \: covered \: in \: k }{|subtopics+time\:windows|}
% \end{equation}


\begin{table*}[!t]
  \small
  \centering
   \scalebox{0.95}{
  \begin{tabular}{llll}\toprule
    %\multicolumn{2}{c}{} & \multicolumn{3}{c}{\textbf{GRD}} && \multicolumn{3}{c}{\textbf{APX}} && \multicolumn{3}{c}{\textbf{OPT}}\\
    %\cmidrule{3-5} \cmidrule{7-9}\cmidrule{11-13}
    \multicolumn{1}{l}{\textbf{Measure}} &&& \multicolumn{1}{l}{\textbf{Formula}}  \\ \midrule

    \textsc{Tia-NDCG}$_k$ &&&  $\sum_{t} P(t|q) \sum_{c} P(c|q) NDCG(q,k|c,t)$ \\ \\
    \textsc{Tia-ERR}$_k$  &&& $\sum_{r}^{k} \frac{1}{r} \sum_{t} P(t|q) \sum_{c} P(c|q) rel(r|c,t) \prod_{i=1}^{r-1} (1-rel(i|c,t))$ \\ \\
    \textsc{Tia-AvgPrecision}$_k$ &&& $\sum_{k=1}^{n} (\textsc{Tia-Precision}_k * rel(k))/|R|$ \\ \\
    \textsc{Tia-MAP}$_k$ &&& $\sum_Q \textsc{Tia-AvgPrecision}_k / |Q|$ \\ \\
    \textsc{T-Sbr}$_k$ &&& $\frac{ |subtopics(S)|\:  +\:  |timeWindows(S)| }{|subtopics(q)|\:  +\:  |timeWindows(q)|}$ \\

    \bottomrule
  \end{tabular}}
  \caption{Time-Aware Effectiveness Measures}
  \label{tab:measures}
\end{table*}
%\vspace{-4mm}


% \begin{equation}
% \textsc{Tia-NDCG}_k = \sum_{t} P(t|q) \sum_{c} P(c|q) NDCG(q,k|c,t)
% \end{equation}

% \begin{equation}
% TIA-ERR = \sum_{r}^{n} \frac{1}{r} \sum_{t} P(t|q) \sum_{c} P(c|q) rel(r|c,t) \prod_{i=1}^{r-1} (1-rel(i|c,t))
% \end{equation}

% \begin{equation}
% TIA-AvgP(q)_k = \frac{\sum_{k=1}^{n} (TIA-Precision(k) * rel(k))}{\# relevant\:documents} 
% \end{equation}

% \begin{equation}
% TIA-MAP = \frac{\sum_Q TIA-AvgP(q) }{|Q|}
% \end{equation}

% \begin{equation}
% SBR_k = \frac{\# subtopics \: covered \: in \: k\:  +\:  \# time \: windows \: covered \: in \: k }{|subtopics+time\:windows|}
% \end{equation}


\subsection{Time-Aware Subtopic Recall}
\label{sec:ta-sbr}
\paragraph{}
To accurately measure the historical value of a result set we need a metric that models the coverage of important time windows and subtopics. It is fair to consider the equal importance of each subtopic although doing the same for time would not provide an accurate way to discriminate. For example, it is safe to assume that a user searching for the history of the 9/11 attacks would prefer to get relevant documents from the year 2001 rather than 2007. Hence it desirable to reward result lists that rank relevant documents from important time periods. We introduce a variant of subtopic recall called time aware subtopic recall. This measure's discriminatory power comes from direct modeling of time windows as bursts rather than new subtopics like in our adaptation of subtopic recall. Each time window $t$ is given a burst weight $P(t|q)$ just like each subtopic. 

%\note{Remove the last 2 equations and cleanup description to reflect only 1 equation i.e. Equation~\ref{eq:tasbr}}

\begin{definition}
The time-aware subtopic recall at k, \textsc{Tia-SBR}$_k$, is defined as
% \begin{equation}
% \label{eq:tasbr}
% \textsc{Tia-SBR}_k =  \alpha  \textsc{SBR}_k +  (1-\alpha) burstrecall_k
% \end{equation}

\begin{equation}
\label{eq:tasbr}
\textsc{Tia-SBR}_k =  \alpha \, \underbrace{\sum_c^{C \in S} P(c|q)}_{\textsc{Sbr}_k} \,\,+ \,\, (1-\alpha) \, \underbrace{\sum_t^{t \in S} P(t/q)}_{burst \,\,recall \,\,at \,\,k}
\end{equation}
\end{definition}

where $\textsc{Sbr}_k$ is a weighted interpretation of subtopic recall. Considering equal importance for all subtopics leads to $P(c|q) = \frac{1}{|C|}$ which is the standard subtopic recall used for diversity evaluation. Intent aware measures will just favor algorithms that can cover more subtopics. Adding time awareness to these measures will have favorable results for algorithms that select important time periods and subtopics. Hence by combining time awareness and intent awareness in the standard measures we are able to determine just how well an algorithm performs for a historical query intent. We also choose \textsc{Tia-Sbr} as the measure of choice for the historical diversification task. 

