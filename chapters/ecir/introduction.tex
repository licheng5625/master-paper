\section{Introduction}

% \todo{
% 	\begin{enumerate}
% 	  \item Historical Query Intent
% 	  \item News collections: each article has clean and well defined aspects
% 	  \item Concept of diversity: both time based and aspect based
% 	\end{enumerate}
% }

From our study of scholarly access of web archives, we found that keyword search is the primary access method. Our findings also show that ranking search results is a problem faced by scholars at the British Library. Users are interested in a getting an overview of search results not only across aspects of the query but also across time. Keyword queries with an inherent historical intent over longitudinal text corpora like web archives are interesting for the scholars like historians, social scientists and journalists. While searching documents published over time, a key preference as we have seen is to retrieve documents which are from the important aspects from important points in time. To this extent, we introduce the notion of a \emph{Historical Query Intent} and define an aspect-time diversification problem over archives. 

In particular, we chose news archive since we have reliable publication dates and a wider time span when compared to some of the standard web crawls available for experimentation. Also since we do not address the problem of multiple versions of the same page, news archives are the ideal sample to test our work.

Consider \textbf{Case 3} from the previous section. Lets say the scholar interested in the history of \textsc{Rudolph Giuliani}, the ex-mayor of New York City in the time interval between 1987 to 2007. She is not only interested in the important facets like \texttt{mayoral campaigns, his mayorality, race for senate} and \texttt{his efforts during 9/11}, but she is interested in articles which cover these aspects when they were important. News articles of historical interest can be classified into breaking news, opinion pages, or summary and reflective pieces. Although reflective or summary articles might mention important aspects, events and news they might always belong to the time of interest of the aspect in question. Thus, presenting a news article about Giuliani's efforts for 9/11 during 2001-2002 is deemed more interesting than articles from other time periods. Similarly, articles talking about the mayoral campaigns in 1989, 1993 and 1997 (the years when the mayoral elections were held) are more effective than documents covering this aspect in 2007. We introduce the notion of such historical query intents or \emph{HQI} which aims to diversify search results by explicitly taking into account the aspects to which the news articles belong alongwith the time of importance of the aspect.

In this chapter, we present a novel algorithm, \textsc{HistDiv}, that explicitly models the aspects and important time windows to which the results of the query belong. We also propose a new metric \textsc{Tia-Sbr} to evaluate the effectiveness of methods intending to solve it. We test our methods by constructing a test collection based on \emph{The New York Times Collection} with a workload of 30 queries assessed manually. Our experiments show that we outperform all the competitors in most of the measures, and remain competitive in a select few.



%To understand the present we must first look into the past. Today, we look to the web to give us answers although the reliability of the documents on the web is questionable. News websites offer a more reliable report on past events. News archives are used extensively by journalists while researching new articles. When trying to unearth the history of Rudolph Giuliani, the ex-Mayor of New York, a journalist would query a news archive search system with the keywords: rudolph giuliani. Using state of the art retrieval models, he will find very relevant docuemts to his query although there is no guarantee that he will get documents from each of his mayoral campaigns and his senate campaign. The task of finding relevant articles in a collection for a given topic is an ad hoc retrieval task as defined by TREC and it is clear that this retrieval task is not suited for a historical intent. Search result diversification is a technique that is used to minimize redundancy in search results which leads to a larger number user intent interpretations or query facets being satisfied. Most algorithms follow a greedy approach which selects documents at each step based on the documents already added to the final ranked list at that point. Historical search intents can be modeled as a type of diversification problem akin to diversifying the facets of an underspecified user intent where in this case facets are important historical subtopics. 

%Lets look at historical query intents a little closer with an example. Consider a user intent: I want to the know history of Rudolph Giuliani, the ex-mayor of New York City, from 1987 to 2007. For this topic some possible facets are: his mayoral campaigns, his mayorality, race for senate and his efforts during 9/11. Standard aspect based diversification techniques will be able to discriminate between these 2 facets but will still not guarantee that we get documents from each of his mayoral capaigns in 1989, 1994 and 1997. This is simply because standard search result diversification techniques are based on aspects and not time. The different facets are represented as aspects of a given query and they try to rank documents to diversify these aspects. For a historical intent though, time is essential and should also be considered as a dimension for diversification. Earlier work on diversification in timestamped document collections~\cite{lmtd} focuses only on diversifying time. Time is considered as a set of disjoint windows. To satisfy a historical user intent though a list of k documents should cover important aspects at important time periods. Showing a user documents from an important time period will also help him understand the history of the topic better. This approach will work for historical search intents if subtopics occur in disjoint time periods but it will not be able to account for diverse aspects within a time window. For example Rudolph Giuliani worked for recovery efforts during 9/11 and was dealing with a bitter divorce as well in 2001. If we consider a time window size of 1 year then it is highly likely that in a reasonable k documents we will cover either his work during 9/11 or his divorce. 

%In general, historical user intent is represented as: I want to know the history of topic x from time $t_1$ to $t_2$. Let the subtopics of topic x be a,b,c and the time span of the result set is 1900 to 2000. The important time periods within $t_1$ and $t_2$ are identified as 1995, 1945 and 1905. Then our ideal result set should cover subtopics a, b and c at their respective important time periods. It is possible that each aspect can be more or less important for a given time period as well. Keeping that in mind we define the task of historical retrieval in archives as the selection of top k documents that cover the most important aspects at the most important time periods. We consider the task to be a time and aspect diversification problem. The modeling of time in relation to an aspect and vice versa explicitly are key to our attempt at providing a retrieval model for this task. In this paper we introduce a novel retrieval task based on historical user intents. We define an appropriate test collection using a news archive and also determine a query workload for the evaluation. We extend existing diversity metrics to incorporate time and also define a new measure to estimate the performance of a retrieval model for this task in particular. We introduce ASPTD, a novel retrieval model designed for historical retrieval tasks. An interesting outcome of this work is also the adaptation of existing diversification methods to include time. We also discuss the merits and demerits of each competitor and also the improvements gained by using ASPTD when presenting the results.

% \bibliographystyle{abbrv}
% \bibliography{bibtex/historical_search}