\section{Related Work}
\vspace{-2mm}
One of the first attempts to incorporate time in search result ranking was suggested by Li et al. \cite{li2003time} who used temporal language model approach where time and term importance are handled implicitly. Various approaches have been suggested that consider time more explicitly. Identifying the time period most useful for query expansion~\cite{kanhabua2010determining,choi2012temporal} has been shown to improve result ranking. Ranking for recency on the other hand considers the freshness of a document when ranking \cite{dong2010towards}. 

For queries which are temporally ambiguous, temporal diversification is applied. Burst detection is first used to find the possible periods of interest for the query, also known as the query's temporal profile\cite{diaz2004using,peetz2012adaptive}, followed by selecting documents from these bursts. Berberich et al. in~\cite{lm+t+d} proposed a diversification model which considers time windows as a set of intents for a query while modeling the importance of each intent as the weight of its burst. In traditional aspect-based diversification tasks like~\cite{clarke2009overview,clarke2011nist}, intent importance is considered static over time. However, intent importance was shown to vary across time; thus affecting the diversity evaluation of queries issued at different time points \cite{zhou2013impact}. Keeping this in mind, Kanhabua et al.\cite{ecir/NguyenK14} consider the time at which the query is issued to diversify intents based on their temporal significance at that time. Their approach also explicitly models time and aspects, although latent, but rewards recency. \textsc{HistDiv}, on the other hand, is query-time agnostic, since it is intended for historical search, and seeks to diversify documents based on both time and aspects. 

Diversity in search (both explicit and implicit) has seen a rich body of literature lately in~\cite{dang_term_2013,carbonell1998use,agrawal_diversifying_2009,santos2010exploiting,Carterette:2009:PMR:1645953.1646116,zhu_learning_2014,liang_fusion_2014,dang_diversity_2012}. However, none of them take time into account or model the historical information intent.

Work on temporal test collections for information retrieval has seen limited attention. The work which comes closest to our test collection is~\emph{Temporalia}\cite{joho2014ntcir} however, our query intents are different and moreover their dataset does not have the temporal spread required for our historical queries.

%Temporal information retrieval has its own diversification of problem. Temporal diversification algorithms like \cite{lm+t+d} take the first view on diversification and aspects are explicitly modeled as time windows. A temporally diverse result set tries to maximize the coverage of time periods. Temporal diversity is also a desireable trait in reccomender systems \cite{lathia_temporal_2010}. \cite{mdiv} introduces a general framework for combining differnet dimensions into a single diversity problem. Our retrieval model takes after this multidimensional approach but introduces a novel utility and discounting function for historical search. Diversity based retrieval models, in general, are evaluated against test collections developed for the TREC diversity task (citations?) with standard intent aware measures such as intent aware ERR \cite{Chapelle:2009:ERR:1645953.1646033}, NRBP \cite{Moffat:2008:RPM:1416950.1416952} $\alpha$-NDCG(citation) and subtopic recall. We modify the standard measures to incorporate time as well due to the nature of historical search. The objective is to not only measure the performance with respect to intents but intents from relevant and important periods of time. Another point to consider is the difference between historical user intents and temporal user intents. Historical user intents are meant to be query time agnostic. Temporal retrieval models like \cite{ecir/NguyenK14} take as input not only the query but also the time at which the query is issued. The query issue date then affects the weight of the various intents related to the query. 