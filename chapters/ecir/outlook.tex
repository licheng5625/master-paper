
\section{Conclusion \& Outlook}
\vspace{-2mm}
In this paper we introduced the notion of a historical query intent over longitudinal news collections like news archives. We proposed a new evaluation metric \textsc{Tia-SBR} to evaluate its performance and built a new temporal test collection based on 20 years of the \emph{New York Times} collection. We introduced \textsc{HistDiv} which shows improvements over temporal and non-temporal methods for most of the time-aware diversification methods. We also outperform all competitors in \textsc{Tia-SBR} showing the suitability of our approach for historical query intents. We observe that \textsc{HistDiv} works well for topics which have aspects that span accross multiple time intervals and have fluctuating importance at different times. It trades-off nicely between important aspects and important times which we percieve as important in historical search. \textsc{HistDiv} does not perform quite as well for queries which only one dominant aspect at a certain time window. This opens up exciting future work opportunities to automatically identify queries of different historical intents and evaluate them accordingly.


%In the future we would like to not only improve HistDiv but also experiment with the automatic selection of retrieval models based on the temporal profile of the query among other factors. Adding extra dimensions of diversity, apart from aspect and time, for historical query intents could also lead to interesting results.


%In this paper we introduce a new type of query intent called historical query intent. A historical query intent refers to a topic for which the user wants to find documents that cover various aspects of the topic's history. We model this intent as 2 dimesnional diversification problem: aspect and temporal diversity. To evaluate such intents we develop a test collection based on the New York Times 1987-2007 news archive and also introduce time aware versions of existing diversity measures. A new measure TA-SBR is introduced for specifically measuring the historical coverage of a result set. We present our method ASPTD which diversifies aspects with respect to time and time periods with respect to aspects covered. The experiments are conducted against 6 state of the art competitors from the diversity literature: MDIV, LM+T+D, IA-SELECT, Temporal IA-SELECT, PM2 and Temporal PM2. We show that ASPTD is superior to our competitors is recall and user centric measures. A major contribution of this work apart from ASPTD is the test collection and judgements tailored for historical query intents which is made available to the public. 

%\paragraph{}
%Since, to the best of our knowledge, this is the first attempt at historical query inents there is much potential for further research. As alluded in the experiments section, ASPTD performs well overall but is lagging behind when the topic implicitly refers to a small time window. More experiments with varying window sizes set manually can also shed some new insights. It will be interesting to know how state of the art burst detection techniques can help improve our method. A point that we do not consider in this work is temporal references within a document. We rely on the publication timestamp as an idicator for time but for some articles, like reflective pieces which consolidate the facts of a past event, this can be misleading. As a consequence of this, another constraint that we make in the test collection is the use of subtopics from only within the time span of the document although if we start using temporal references from the content of the document it is conceivable that we can go outside this time window even if the collection itself does not stretch into the past. We have also not tried to compare ourselves with algorithms which rely on query logs which leads us to an interesting sub problem of trying to generate a reasonable query log over time that can be mined for historically relevant subtopics. Finally we hope to continue developing the test collection in collaboration with our experts and historians to improve both the quantity and quality of data. 
