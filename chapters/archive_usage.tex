\chapter{Background and Related Work} % (fold)
\label{cha:Related_Work}
\section{Twitter}
Twitter is a mircoblogging service. Now there are more than 140 million active users\footnote{https://blog.Twitter.com/2012/Twitter-turns-six}. User can publish short messages within 140 characters aka tweets.
\subsection{Retweet}
A retweet is re-posting of a tweet by other users. One way of retweet is using RT at the beginning of the tweet which is retweeted. The other way is using "retweet button" which is officially launched by Twitter after 2015. The difference between these two retweet is tweets are retweeted by "retweet button" can't be searched by Twitter's searching interface, but manually retweets with RT keyword can. So in our work retweet means the tweets are retweeted manually. The number of how many times of this tweet has been retweeted is showed behind it.

\subsection{Mentions}
Mentions are in form like "@username" which are added in the text of tweet. The users are mentioned will receive the notification of this tweet on their homepage.
\subsection{Hashtags}
Mentions are in form like "\#topic". It means this tweet belongs to some topics.

\subsection{Favorite}
Favorites means how many users like this tweet. It is showed on the interface of Twitter
\subsection{Verified User}
Verified User means this account of public interest is authentic by Twitter. It is showed by a blue icon behind the name of the poster.
\subsection{Followers}
The followers of a user are accounts who receive this user's posting. The total number of followers can been seen in the profile of poster.
\subsection{Following}
The following are other accounts who follow this user. The total number of following can been seen in the profile of poster.
\subsection{Twitter API}
Twitter API is provided by Twitter\footnote{https://dev.twitter.com/overview/api} for developer. But the search API only return a sampling of recent Tweets published in the past 7 days\footnote{https://dev.twitter.com/rest/public/search}. We need the full stories of the events, so we crawled the data directly from the searching interface\footnote{https://twitter.com/search-home}.

 \section{Definition of Rumor}
 The definition of rumor in our work is unverified information spreading on Twitter over time. It is a set of tweets including the the sources, retweets and debunking tweets. 
 \subsection{Rumor Event}
 And if a rumor didn't widely spread and it could be harmless. So we more focused on the rumors which are widely spread and contain one or more bursty pikes during propagation. We call it "rumor event".
\subsection{Time Period of an Event}
\label{sec:Time_Period_of_an_Event}

The beginning of a rumor is hard to definition. And every formal work didn't mention how to define the beginning of one rumor event. One rumor may be created serval years ago, but it can be triggered by other events and quickly spread.
For example, a rumor\footnote{http://www.snopes.com/robert-byrd-kkk-photo/} claimed that Robert Byrd was member of KKK. This rumor has been circulating in Internet for a while, as shown in figure(wenxian) that almost every day there are several tweets. But this rumor was triggered by a picture between him and Hillary Clinton.We defined the hour which has the most tweet's volume as $t_{max}$ and the first tweet before $t_max$ within 48 hours we defined it as the beginning of this rumor event $t_b$. And the end time of events we defined as  $t_{end}=t_b+48$.


\section{Machine learning } % (fold)
\label{sec:Maschine_learning}
\subsection{Machine learning Overview} % (fold)

% Why are archives valuable
 Machine learning covers vast numbers of algorithms and has been successful applied in different field. The challenges of ML are finding the best model which is suitable for this task, fitting the parameters and selecting the features. 
 
Normally we split the ML methods into Supervised learning, Unsupervised learning and Reinforcement learning\cite{russell2003artificial}. 

The supervised learning is the most popular method of ML. It needs a set of inputs and a set of desired outputs. And the algorithm will learn to produce the correct output based on the new input. 

The unsupervised learning needs a set of inputs but no outputs. The algorithm will generate the outputs like clusters or patterns. The unsupervised learning task can be used for example when people can't label the outputs. 

\subsection{random forest (RF)} % (fold)
Classification is a supervised data mining technique. Our work can be considered as a classification task. And the classification model is random forest.

RF is an algorithm of supervised learning which developed by Leo Breiman\cite{breiman2001random}. It's built by a set of classification trees\cite{breiman1984classification}. Each tree is trained by a small bootstrap sample of training set and while prediction each tree votes one single candidate. By taking the majority vote RF can produce the result of prediction. 

Because RF uses a random subset of features instead of the best features in every node, so it can avoid the overfitting\cite{breiman2001random}. 

Another benefit of RF is that it can return the features' importance. RF is built up by a subset training data and the data we didn't selected we call it out-of-bag (OOB) data and we can validate the model by using OOB data we got OBB error $E_{oob}(G) = \frac {1}{N} \sum_{n=1}^{N}err(y_n,G_{n}^-(X_n))$ with $X_n$ are features only in OOB. At last we get the importance of feature by permuting one feature to random numbers. $importance(i)= E_{oob}(G)-E_{oob}^{p}(G)$ where $E_{oob}^{p}(G)$ is the OBB error after permuting a feature. We use this method to rank the fearers in section(wenxian).
\section{Credibility of tweets } % (fold)
Titter has been used for reporting breaking news when emergency events happen like disaster (wenxian). 


\section{Related Work } % (fold)
People study on rumors in psychology for years\cite{allport1947psychology}.
But recently the detection misinformation on microblogging becomes a trending researching topic. Researcher first began from several special events like natural disasters(wenxian). These results are not general enough to other type of rumors. Carlos Castillo researched the information credibility on Twitter. 



